\documentclass[12pt]{article}
\begin{document}
    
    \section{Chemistry}\label{chemistry}
    \emph{notes from Atkin's `Chemical Principles'}

    \begin{itemize}
        \item A metal conducts electricity and is malleable and ductile \item
          A metalloid can have the appearance of a metal or non metal but behave
          likes its opposite
        \item The definition for metals is not very precise but the distinction
            is commonly made 
    \end{itemize}

    \subsection{Mixtures and solutions}\label{mixtures-and-solutions}

        \begin{itemize}
        \item
          The formation of a mixture is physical change whereas the formation of
          a compound is chemical change
        \item
          In a solution (homogeneous mixture) there is typically a dominant
          substance (solvent) with other substances present (the solute)
        \item
          Crystallization is the process by which a solute slowly comes out of a
          solution as crystals
        \item
          Aqueous solutions are solutions in which the solvent is water.
        \end{itemize}

    \section{Reactions}\label{reactions}

        \begin{itemize}
        \item There are 3 really common reactions in nature: precipitation,
            oxidation-reduction and acid-base.
        \item Oxygen is the most electronegative element (apart from Fluorine)
            in the periodic table. It is highly reactive and normally occurs in
            as a minus to ion because it want's 2 electrons to complete it's
            outer shell and it wants them really bad.
        \item Hydrogen can play a role as an oxidiser and a reducer.
    \end{itemize}

    \subsection{Precipitation reaction}\label{precipitation-reaction}

        \begin{itemize}
        \item
          The extent to which any substance may be dissolved in water is
          quantitatively expressed by \emph{solubility}. It's a measure of
          concentration of a substance in water at specific conditions.
        \item
          When the concentration of a substance in water exceeds it's solubility
          for those conditions precipitation reaction occur.
        \item
          The forming of a solid is the \emph{driving force} of this reaction.
        \end{itemize}

    \subsection{Acids}\label{acids}

        \begin{itemize}
        \item Arrhenius acid: a compound that contains hydrogen and releases
            hydrogen ions in water. \item
          Arrhenius base a compound that produces hydroxide ions in water.
        \item Acid and base reactions could be considered proton transfer reactions,
          redox reactions are electron transfer.
        \item The pH level is the hydrogen ion concentration in a solution. The
          higher the hydronium (hydrogen ion) concentration the lower the pH.
          \(pH = -log[H_3O^+]\) a change of 1 of pH is factor of 10 change for
          the actual hydrogen ions.
        \item Hydronium are hydrogen ions (so protons) in a water solution. Any free
          roaming proton in a water solution will be bonded to H2O making H3O (a
          positive ion).
        \item A \emph{salt} is an ionic compound formed during neutralisation.
        \end{itemize}

\subsection{Atomic Structure}\label{atomic-structure}

\item
  \begin{quote}
  Planck's hypothesis implies that radiation of frequency \(v\) is
  generated only when an oscillator of that frequency has acquired the
  minimum energy required to start oscillating and then ejects it as a
  packet of electromagnetic radiation of energy \(hv\).
  \end{quote}
\item
  The photoelectric effect fortifies the theory that elements have
  activation energies that must be overcome before EM radiation is
  emitted (or, before the electrons (and other particles?) in the atom
  become oscillators of a given frequency
\item Kinetic energy of the emitted electron then varies linearly with
  frequency
\end{itemize}

    \subsubsection{Building up principle}\label{building-up-principle}

        \begin{itemize}
        \item In the ground state of a many-electron atom the electrons occupy
          orbitals in such a way that the total energy of the atom is a minimum
        \item This would in theory mean all the electrons occupy the 1s orbital but
          the Pauli exclusion principle prevents this
        \item \begin{quote}
          No more than two electrons may occupy any given orbital. When two
          electrons do occupy one orbital, their spins must be paired
          \end{quote}
        \item In general only valence electrons can be lost or gained in chemical
          reactions so atoms can be thought of a core shell, in the case of
          lithium a \(1s^2\) helium shell surrounded by \(2s^1\) valence shell
        \item In the case of the 2p subshell if we have 6 electrons, how do we know
          which orbital the 6 electron should occupy?
        \item Electrons repel each other less in different orbitals. So the less
    energy option is to take up a parallel spin in an empty 2p orbital
        \end{itemize}

\subsubsection{Ionization energies}\label{ionization-energies}

    \begin{itemize}
    \item
      The \textbf{first ionization energy} is the minimum energy required to
      remove an electron from a neutral atom in the gas state. The
      \textbf{second ionization energy} is the energy needed to remove an
      electron from a cation in the gas state.
    \item
      The second ionization energy is larger than the first, it takes more
      energy to remove an electron from a cation
    \end{itemize}

 {minerals}{%
\subsection{Minerals}\label{minerals}}

\begin{itemize}
\item
  A solid chemical compound with a well defined structure, occurring
  naturally
\item
  There's different groupings of minerals depending on composition
  e.g.~sulfides, oxides, carbonates, silicates
\end{itemize}

 {solids}{%
\subsection{Solids}\label{solids}}

\begin{itemize}
\item
  A solid material whose constituents are arranged in a highly ordered
  structure forming a crystal lattice (an array of discrete points)
\item
  A crystalline solid has long range order. Locally in a amorphous solid
  there might be a pattern but it's not indefinitely repeated
\item
  Solids can be classified by the bonds that hold their structure. The s
  and d block elements are often defined as metallic solids because they
  are held together by cations amongst a sea of electrons

  \begin{itemize}
  \item
    \begin{quote}
    The solid phases of three quarters of all elements are metals
    \end{quote}
  \item
    In metals the valence electrons can be shared with the hole
    structure and move relatively freely (by having a light interaction
    with everything?)
  \item
    Forming an `electron sea' all the ions crumple up because there's an
    interaction with electrons in all directions, this creates the close
    packed kind of model. Atoms are stacked in either a hexagonal or
    cubic arrangement, the maximum coordination number being 12
  \item
    A cubic close packed unit cell can also be called a face-centered
    cubic structure
  \end{itemize}
\item
  Network solid are those solids that are covalent bonded to eachother
  throughout the solid
\end{itemize}

 {bonding}{%
\subsection{Bonding}\label{bonding}}

\begin{itemize}
\item
  All cations are smaller than the `parent' element as they lose
  electrons reducing shell radius
\item
  The opposite happens for anions as the interact with eachother when
  added
\item
  Metals on the right side of the periodic table want to reach the noble
  gas state to their right (gaining electrons) so they're more likely to
  be anions and metals on the left want to lose electrons to get to
  noble gas configuration on the left (more likely to be cations)
\item
  As you move across a period, ionization energies increase as
  electrostatic with the nucleus increases
\item
  No bond is totally ionic or covalent. Which way it tends is based on
  electronegativity.
\item
  If the difference in electronegativity is small the bond will be
  largely covalent.
\item
  A bond is ionic or covalent because that's a good approximation of
  what it probably is.
\item
  What does it mean to say that a certain bonding couple is
  energetically favourable?

  \begin{itemize}
  \item
    If we think of two atoms of a particular elements floating in some
    hypothetical free space.
  \item
    There is an electrostatic interaction between these atoms.
  \end{itemize}
\end{itemize}

 {vsper-model}{%
\subsection{VSPER Model}\label{vsper-model}}

\begin{itemize}
\item
  Lewis structures are useful in representing no. of lone pairs in a
  molecule and the linkages between atoms. It doesn't say anything of
  the shape of molecules
\item
  The VSPER model adds more rules and represents molecular shapes

  \begin{itemize}
  \item
    Regions of high electron concentration repel eachother, moving far
    apart but maintaing their distance to the central atom. Lone pairs
    not considered yet on the central atom (for some reason)
  \item
    Only the positions of atoms are considered in determining the shape
    of the molecule but all effects from lone pairs and bonds are also
    included in this description
  \end{itemize}
\item
  \begin{quote}
  If there are no lone pairs on the central atom (an AX n molecule),
  each region of high electron concentration corresponds to an atom, and
  so the molecular shape is the same as the electron arrangement.
  \end{quote}
\item
  A polar bond is one where the atoms do not have equal `hold' of the
  electron they're sharing. A polar molecule is one with a non zero
  dipole moment.
\end{itemize}

 {valence-bond-theory}{%
\subsection{Valence Bond theory}\label{valence-bond-theory}}

\begin{itemize}
\item
  takes the wave nature of electrons into account. The previous two
  models are localized electron models in the sense that a bond and
  electron have a fixed place
\item
  \(\sigma\) bonds for when opposite spins overlap, forming an orbital
  around both atoms. All single bonds are sigma bonds
\item
  \(\pi\) bonds just have a single nodal place on the axis joining the
  nucleus. Kind of like the \(\sigma\) bond turned on its side
\item
  Interference in wave patterns can result in hybrid orbitals
\item
  Process is: Identify valence shell atomic orbitals that contain
  unpaired electrons then allow these orbitals to overlap forming
  \(\sigma\) and \(\pi\) bonds.
\item
  Also, mainly describing bonding between the same types of atom.
\end{itemize}

 {hybridization}{%
\subsection{Hybridization}\label{hybridization}}

\emph{Hybridization also converges towards lower energy levels, why is
this?}

\begin{itemize}
\item
  Hybridization occurs in the creation of methane which is one carbon
  atom surrounded in a tetrahedral structure by hydrogen.
\item
  The bond with 4 hydrogen atoms can be created by excitation of the 2s
  orbital to `promote' one of the electrons to the vacant p orbital.
\item
  The energy released on the creation of the 4 bonds is much greater
  than that of excitation of the 2s state so it is energetically
  favourable.
\item
  This creates a \(sp^3\) orbital. Which yields 4 \(\sigma\) bonds
\end{itemize}

\section{Physical Equilibria}\label{physical-equilibria}

 {vapor-pressure}{%
\subsubsection{Vapor Pressure}\label{vapor-pressure}}

\begin{itemize}
\item
  For a substance, it's vapor pressure is the pressure exerted by it's
  vapor when it's in dynamic equilibrium with it's condensed phase.
\item
  The vapor pressure of a liquid (at a given temp.) is expected to be
  low if it's intermolecular forces are strong.
\item
  Vapor pressure rises with increased temperature.
\item
  The Clausius-Clapeyron equation gives the quantitative dependence of
  vapour pressure on temperature.
\item
  The imbalance between atmospheric pressure and vapor pressure is how
  boiling works.
\end{itemize}

\emph{What is dynamic about vapor pressure?}

\begin{itemize}
\item
  If we put water into a container that is in a vacuum
\item
  If we put a liquid into a sealed container. The liquid will start to
  evaporate as molecule from the surface has enough energy to escape
  into the air above.
\item
  Eventually there are enough molecules in the space above the liquid
  that a molecule that enters that region (or is in that region) loses
  energy and rejoins the liquid phase.
\item
  The vapor pressure is the pressure exerted when these two rates are
  equal.
\item
  As the average speed of molecules reduce, they are more likely to
  clump together.
\item
  So we have two processes operating

  \begin{itemize}
  \item
    At a given temperature for volatile substance in a closed container
    more of it's molecules will enter the vapor state trying to converge
    on some equilibrium. This equilibrium may never be reached by the
    way.
  \item
    In an open system, like water in a beaker, the vaporisation rate is
    never in dynamic equilibrium with the condensation rate. The water
    evaporates away.
  \end{itemize}
\end{itemize}

\subsubsection{Boiling}\label{boiling}

\begin{itemize}
\item
  The \emph{boiling point} is the temperature at which the vapor
  pressure of the liquid is 1 atm.
\item
  Happens when the molecules entering the gas state are not just the
  surface of the liquid, because the surface molecules kind of lift up
  the weight above them allowing the lower levels to start entering the
  gas state.
\end{itemize}

\subsubsection{Freezing and Melting}\label{freezing-and-melting}

\begin{itemize}
\item
  Water freezes when the molecules don't have enough kinetic energy to
  move past their neighbours. They might then not have the energy to
  overcome dipole interactions (or hydrogen bonds?).
\item
  The \emph{freezing temperature}, the temperature at which the solid
  and liquid phases are in dynamic equilibrium. The \emph{normal}
  freezing temperature is whatever this is at 1 atm. This is why
  sometimes you've to go a bit below 0C to fully get solid ice.
\item
  The critical point defines a state at which the substance becomes a
  really dense gas. But not a gas that we can condense to a liquid form.
  Called a \emph{supercritical fluid}.
\end{itemize}

\section{Chemical Equilibria}\label{chemical-equilibria}

\emph{Start to look at dynamic equilibrium that is reached in chemical
reactions}

 {reaction-quotient}{%
\subsection{Reaction Quotient}\label{reaction-quotient}}

\begin{itemize}
\item
  \emph{law of mass action}: over a series of experiments with the same
  relative concentrations for a reaction the same reaction quotient
  appears.
\item
  Each reaction has it's own characteristic equilibrium constant, but
  that value is dependent on the temperature.
\item
  The magnitude of the equilibrium constant is a measure of the extent
  to which the reaction favours the products or reactants.
\item
  Chemical reactions reach a state of dynamic equilibrium in which the
  rates of forward and reverse reactions are equal and there is no net
  change in composition.
\item
  Most reactions are those shifted very slightly to the products.
\item
  Gibbs energy helps us look at the mechanism or why the reaction
  quotient is the way it is.
\item
  At equilibrium further formation of the reactants is not spontaneous.
\item
  The Gibbs energy is dependent on the proportions of the reactants.
  Every chemical reaction tends towards a Gibbs equilibrium
  \(\Delta G = 0\).
\item
  \[\Delta G = \sum nG_m(\text{products}) - \sum nG_m(\text{reactants})\]
\item
  n is a stoichiometric coefficient, a number, not necessarily a mole.
\item
  Equilibrium constant can come in forms of partial pressure,
  concentrations etc.
\end{itemize}

\section{Kinetics}\label{kinetics}

\begin{itemize}
\item
  Average rate of consumption of reactant R is
  \(\frac{\Delta[R]}{\Delta t}\)
\item
  {[}R{]} is the molar concentration. \(\Delta[R]\) is negative because
  reactants are used up in a reaction to create products.
\item
  The \emph{unique average rate} is a way to report the rate of reaction
  without explicitly stating the species were talking about. It involves
  dividing by the stoichiometric coefficient.
\item
  Initial rate, the instantaneous reaction rate at t = 0 for a species
  is proportional to the initial concentration of that species.
\item
  It get's harder to speak about reaction rates as reactants are forming
  into products and vice versa. The initial rate is when no products are
  formed.
\item
  The plot of initial rate vs initial concentration gives a straight
  line. The constant of which is called the rate constant (\(k\)).
\item \(k\) is characteristic of a reaction, or specific to the reaction
  taking place and the temperature at which it takes place.
\item Rate of consumption of a particular reactant = \(k_r[\text{Molar concentration of reactant}]\)

  \begin{itemize}
  \item
    This is a \emph{rate law}. An expression for the instantaneous
    reaction rate in terms of the concentration of the reaction at any
    time.
  \end{itemize}
\item
  In some cases initial rate may not be proportional to initial
  concentration but some order of the initial concentration.
\item
  In many cases the products do not take part in the reaction, so the
  expression for the initial rate is applicable to later stages of the
  reaction when products are present.
\item
  A \emph{zeroth-order reaction} is independent of the reactants as
  longs as there's some of it there.
\item
  \textbf{Rate law's for reactions are determined experimentally}
\end{itemize}

\subsection{Integrated rate laws}\label{integrated-rate-laws}

\begin{itemize}
\item
  If we integrate the rate laws they should give us a value of
  concentration of the reactants (or any species involved in the rate
  law) at a specific time.
\item
  We can use the integral of a first order rate equation as a measure of
  whether a reaction is first order or not.
\item
  If \(ln[A]_t = ln[A]_0 - k _rt\) is linear, then it is a first order
  process.
\item
  A half life for a first order reaction is
  \(t_{\frac{1}{2}} = \frac{ln(2)}{k_r}\)
\item
  The half life has the same value at all stages of the reaction because
  it doesn't depend on concentration.
\end{itemize}

\subsection{Reaction mechanisms}\label{reaction-mechanisms}

\begin{itemize}
\item
  Each \emph{elementary reaction} describes a distinct event in the
  progress of a reaction, often a collision of particles.
\item
  To describe how a reaction takes place, chemists propose a reaction
  mechanism, which is a sequence of elementary reactions on how they
  believe the reactants are transformed into the products.
\item
  The summation of the elementary reactions must give the initial
  reaction. Reaction intermediates can exits, but they must cancel out.
\item
  Elementary reactions are classified by how many molecules are involved
  in the reaction equation. If it's one molecule splitting it's
  unimolecular, two molecules coming together or interacting,
  bimolecular.
\item
  Kinetic information can only support a proposed mechanism (through
  verification of rate laws of reaction mechanism and experimentation)
  it can never actually say it is the mechanism as some other mechanism
  could lead to the same rate law.
\end{itemize}

\subsubsection{Chain Reaction}\label{chain-reaction}

\begin{itemize}
\item
  In chain reactions the reaction intermediary creates more reactionary
  intermediates (normally radicals).
\item
  Normally described with an initiation stage that produces the chain
  carrier or radical in radical chain reactions. Then an equation of
  steps to describe the propagation reactions. Finish with the
  termination process.
\end{itemize}

\begin{itemize}
\item
  Arrhenius tries to describe how reaction rates change with temperature
  using the line equation.
\item
  \[lnk_r = lnA - \frac{E_a}{RT}\]
\item
  Reactions that give a straight line when \(lnk_r\) is plotted against
  \(\frac{1}{T}\) are said to show \emph{Arrhenius behaviour}.
\item
  \(E_a\) is the \emph{activation energy}, because the slope of the
  reaction rate line is proportional to this. The large the activation
  energy the larger the dependence of temperature and the rate constant.
\end{itemize}

\section{Nuclear chemistry}\label{nuclear-chemistry}

\begin{itemize}
\item
  In the earlier 1900's the Curie's propose that perceived radiation
  emitted from blocks of elements is something all atoms do referred to
  as \emph{radioactivity}.
\item
  The existence of the nucleus was unknown at the time. Rutherford
  subjected the radiation to differing magnetic fields when he
  discovered the nucleus he deemed that \(\alpha\) particles (because
  they were positively charged) must be a helium nucleus.
\item
  By measuring the mass of the emission that was attracted to the field
  he found that they were electrons.
\item
  A change in composition of the nucleus is called a \emph{nuclear
  reaction}
\item
  Nuclei that change their atomic structure spontaneously and emit
  radiation are called radioactive.
\item All nuclei with Z 82 are unstable and decay mainly through \(\alpha\)
    emission. For some reason these atoms want to shed protons until the reach
    under 80 or so.
\end{itemize}

\subsection{Nuclear fission}\label{nuclear-fission}

\emph{notes from introductory nuclear physics}

\begin{itemize}
\item
  Enrico Fermi after the discovery of the neutron in the 30's started
  experimenting with bombarding the nucleus with neutrons.
\item
  He found that atoms bombarded in such a weight emitted \(\beta^-\)
  trying to compensate for the addition of the new neutron by creating a
  proton. This led to an attempt to produce heavy elements than uranium
  (the heaviest naturally occurring element).
\item
  The produced atoms seem to be really low down the table (barium) and
  large amounts of energy were released with this neutron capture.
\item
  It was deemed that the bombardment of Uranium causes it to split
  nearly in half, \emph{fission}.
\item
  The total nuclear binding energy increases roughly in proportion to A,
  while the Coulomb repulsion energy of the proton increases faster,
  like \(Z^2\).
\end{itemize}

\section{Organic Chemistry}\label{organic-chemistry}

There are 4 general classes of hydrocarbons:
\begin{itemize}
    \item Alkanes: composed entirely of single bonds and saturated with
        hydrogen. Normal alkanes, the smallest of which is methane. 
    \item  Alkenes: non-saturated with hydrogen, those with a double bond to
        carbon are called alkenes and those with a triple bond \it{alkynes}.
    \item > The inherent ability of hydrocarbons to bond to themselves is known
        as catenation, and allows hydrocarbon to form more complex molecules,
        such as cyclohexane, and in rarer cases, arenes such as benzene. This
        ability comes from the fact that the bond character between carbon atoms
        is entirely non-polar, in that the distribution of electrons between the
        two elements is somewhat even due to the same electronegativity values
        of the elements (~0.30).
\end{itemize}

Naming
\begin{itemize}
    \item Count the carbon atoms in the longest chain
    \item Designate a side chain (say, methane $CH_{3}$ and change from \textit{ane} to \textit{ethyl}
    \item Double bonds are indicated by change \textit{-ane} to \textit{-ene}
    \item count number of links between substituent or designated side chain.
\end{itemize}


\end{document}
