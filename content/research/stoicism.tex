\documentclass[11pt]{article}

% basic packages
\usepackage[margin=1in]{geometry}
\usepackage[pdftex]{graphicx}
\usepackage{amsmath,amssymb,amsthm}
\usepackage{custom}

% page formatting
\usepackage{fancyhdr}
\pagestyle{fancy}

\renewcommand{\sectionmark}[1]{\markright{\textsf{\arabic{section}. #1}}}
\renewcommand{\subsectionmark}[1]{}
\lhead{\textbf{\thepage} \ \ \nouppercase{\rightmark}}
\chead{}
\rhead{}
\lfoot{}
\cfoot{}
\rfoot{}
\setlength{\headheight}{14pt}

\linespread{1.03} % give a little extra room
\setlength{\parindent}{0.2in} % reduce paragraph indent a bit
\setcounter{secnumdepth}{2} % no numbered subsubsections
\setcounter{tocdepth}{2} % no subsubsections in ToC

\begin{document}

% make title page
\thispagestyle{empty}
\bigskip \
\vspace{0.1cm}

\begin{center}
{\fontsize{36}{36} \selectfont \bf \sffamily Stoicism Notes}
\vskip 24pt
{\fontsize{18}{18} \selectfont \rmfamily Conor Redington} 
\vskip 24pt
\end{center}

{\parindent0pt \baselineskip=15.5pt \lipsum[1-4]}

% make table of contents
\newpage
\microtoc
\newpage

% main content
\hypertarget{stoicism}{%
\section{Stoicism}\label{stoicism}}

\emph{Notes on Art of Living and general books around Stoicism}

What's attractive to me about Stoicism is the application of philosophy
to life. First of all, the notion that we have a soul, that we want to
be pure, or that we pursue virtue and try and justify why is worth
questioning. Is it social conditioning that this is the case? If so,
it's some general social conditioning over vast timescales which is
essentially the same thing as `inherently human' or even an `objective
truth'.

\hypertarget{ethics}{%
\section{Ethics}\label{ethics}}

\begin{itemize}
\tightlist
\item
  What is \emph{good} for the Stoics is whatever the rational soul
  desires. As to what this is, how variable it is, I'm not too sure.
\end{itemize}

\hypertarget{physics}{%
\section{Physics}\label{physics}}

\begin{itemize}
\item
  The cosmos, for the Stoics is the large organism of a kind. From it's
  generation (generative principle) it's causes are predetermined.
\item
  This notion is present in Meditations as Marcus tries to integrate
  himself with this cosmos,
\item
  \begin{quote}
  Freeing oneself of this limited first person perspective will free one
  from the emotional turmoil that comes with it.
  \end{quote}
\item
  In the same way, when something is seen as food `for me' or `bad for
  me' elevating oneself to this position allows it's evaluation outside
  of this first person perspective.
\end{itemize}

\hypertarget{art-of-living}{%
\section{Art of Living}\label{art-of-living}}

\href{../books/ArtOfLiving.md}{see more}

\begin{quote}
Philosophy does not promise to secure anything external for man,
otherwise it would be admitting something that lies beyond its proper
subject matter. For just as wood is the material of the carpenter,
bronze that of the statuary, so each individual's own life is the
material of the art of living.
\end{quote}

\begin{itemize}
\tightlist
\item
  Epictetus
\end{itemize}

Seller's describes the aim of his book `The Art of Living' es exploring
``the possibility of a conception of philosophy in which philosophical
ideas are primarily expressed in behaviour, a conception in which
understanding is developed not for its own sake but rather in order to
transform one's way of life, a conception of philosophy that would make
biography not merely incidentally relevant but rather of central
importance to philosophy''.

\begin{itemize}
\tightlist
\item
  Socrates forms this basis or template for philosophy could have been
  viewed to those after his time. That it's aform of mastery over one's
  soul. What interesting is how resonant this becomes. There is a
  faculty of our nature that agrees with virtue. Maybe, from a social
  perspective it's important to have temperance.
\item
  Highlighted in this book is the notion of the different kinds of arts.
  Particularly the art of medicine where you're goal isn't to always
  succeed but be in pursuit of accomplishment. I think this is something
  the Stoics or at least popular culture Stoicism would agree with.
\end{itemize}

\end{document}
