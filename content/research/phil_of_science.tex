\documentclass[11pt]{article}

% basic packages
\usepackage[margin=1in]{geometry}
\usepackage[pdftex]{graphicx}
\usepackage{amsmath,amssymb,amsthm}
\usepackage{custom}

% page formatting
\usepackage{fancyhdr}
\pagestyle{fancy}

\renewcommand{\sectionmark}[1]{\markright{\textsf{\arabic{section}. #1}}}
\renewcommand{\subsectionmark}[1]{}
\lhead{\textbf{\thepage} \ \ \nouppercase{\rightmark}}
\chead{}
\rhead{}
\lfoot{}
\cfoot{}
\rfoot{}
\setlength{\headheight}{14pt}

\linespread{1.03} % give a little extra room
\setlength{\parindent}{0.2in} % reduce paragraph indent a bit
\setcounter{secnumdepth}{2} % no numbered subsubsections
\setcounter{tocdepth}{2} % no subsubsections in ToC

\begin{document}

% make title page
\thispagestyle{empty}
\bigskip \
\vspace{0.1cm}

\begin{center}
{\fontsize{36}{36} \selectfont \bf \sffamily Philosophy of Science Notes}
\vskip 24pt
{\fontsize{18}{18} \selectfont \rmfamily Conor Redington} 
\vskip 24pt
\end{center}

{\parindent0pt \baselineskip=15.5pt \lipsum[1-4]}

% make table of contents
\newpage
\microtoc
\newpage

% main content
\hypertarget{philosophy-of-science}{%
\section{Philosophy of Science}\label{philosophy-of-science}}

\emph{Book notes from Ladyman}

``\emph{Well you can argue all you like but I am going to carry on
believing the scientists and not the people who tell me that the world
will end and that I had better repent, and give them all my money. By
induction, I know that they are very probably wrong, and the fact that I
can't convince}''

\hypertarget{bacon}{%
\subsection{Bacon}\label{bacon}}

\emph{The problem with deductive arguments is that it cannot say more
than is implicit in the premises.}

\begin{itemize}
\item
  bacon has this idea that we should do all we can to stop fooling
  ourselves. That we should be mindful of the Iodls of the mind, tribe
  etc.
\item
  Experiment allows counterfactuals. Allows us to `torture nature for
  her secrets'.
\item
  Bacon also advocates this search for the forms of natural phenomenon
  as opposed to the final cause, the ultimate cause.
\item
  \begin{quote}
  Bacon explicitly urged that teleological reasoning be confined to the
  explanation of human affairs where it is legitimate since people are
  agents who act so as to bring about their goals.
  \end{quote}
\item
  Finally, Bacon argues that when we've two rival equally explanatory
  theories we must devise a `pejorative instance', an experiment to
  result in two outcomes based on what each theory might predict
  (`crucial experiments').
\end{itemize}

\hypertarget{induction}{%
\section{2. Induction}\label{induction}}

\begin{itemize}
\tightlist
\item
  Inductivism as Bacon uses it is generalising from a collection of
  instances to some conclusion.
\end{itemize}

\hypertarget{problem-of-induction}{%
\subsection{Problem of Induction}\label{problem-of-induction}}

\begin{itemize}
\tightlist
\item
  Hume makes the distinction between relations of ideas and matters of
  fact.
\item
  Relations of ideas can be proven to be proved by deduction because its
  negation will imply a contradiction.
\item
  If you can proved that its negation is inconsistent with other stuff
  you know.
\item
  In my thinking, because matters of fact, or knowledge that is
  justified by sense perception is not a priori knowledge than we can't
  prove it as true.
\end{itemize}

Apparently this is similar to what Kant talks about. Positivist
developed this in the early 20th century to the notion that if what you
say can't have implication for what we might observe in the future, then
it's not knowledge.

\begin{itemize}
\tightlist
\item
  Hume claimed that all reasoning that goes beyond past and present is
  based on cause and effect.
\item
  It is \emph{causal relation} that connects ideas that have no logical
  relation. It is this relation that must be understood. There is no way
  to
\end{itemize}

\begin{quote}
We cannot tell that fire will burn us or that gunpowder will explode
without trying it out because there is no contradiction in supposing
that, for example, the next fire we test will not burn but freeze a hand
placed in it.
\end{quote}

\begin{itemize}
\tightlist
\item
  Importantly, belief in the relation of cause and effect is also based
  on believing the future will be the same as the past.
\end{itemize}

\hypertarget{cause-and-effect-for-hume-via-ladyman}{%
\subsection{Cause and effect for Hume (via
Ladyman)}\label{cause-and-effect-for-hume-via-ladyman}}

\begin{itemize}
\tightlist
\item
  Events of type A precede events of type B in time.
\item
  Events A are constantly conjoined in experience with events B.
\item
  Event A lead to the expectation of event B.
\end{itemize}

Some would say there is a necessary connection between the two events
rather than just this pattern that seems to hold. By Occams razor we
would choose Hume's view.

\hypertarget{problems}{%
\subsection{Problems}\label{problems}}

\begin{quote}
It should be noted that various combinations of strategies 1, 5, 6, 7, 8
and 9 are the most popular in contemporary philosophy. Hence, someone
might argue that Hume's argument shows us not that induction is
irrational but that something is wrong with his reason- ing (the
sophisticated version of strategy 1), that what is wrong is that his
account of our inductive practices is too crude (strategy 5), that our
inductive practices really depend on inference to the best explanation
where the explanations in question involve the existence of causal
relations or laws of nature (strategies 6 and 7), and that inference to
the best explanation cannot be justified in a completely
non-question-begging way, but then no form of inference can (strat- egy
8). To this we might add that we only ever end up with a high degree of
belief rather than certainty and that this is the best we can achieve
and is, moreover, psychologically realistic (strategy 9).
\end{quote}

\begin{itemize}
\tightlist
\item
  I think the question begging notion of induction (and deduction) is
  interesting. That you must accept them as your mode of reasoning to
  accept a proof of them. In some sense, maybe we do have to `believe in
  science'.

  \begin{itemize}
  \item
    \begin{quote}
    deductive inference is only defensible by appeal to deductive
    inference \ldots{} someone believes some proposition, p, and they
    also believe that if p is true then another proposition q follows,
    and so they infer q. What could they say to someone who refused to
    accept this form of inference? They might argue as follows; look,
    you believe p, and you believe if p then q, so you must believe q,
    because if p is true and if p then q is true then q must be true as
    well. They reply, `OK, I believe p, and I believe if p then q, and I
    even believe that if p is true and if p then q is true then q must
    be true as well; however, I don't believe q'. What can we say now?
    \end{quote}
  \end{itemize}
\item
  The probabilistic or notion of inference to best explanation also
  seems reasonable as a counter.
\end{itemize}

\begin{center}\rule{0.5\linewidth}{\linethickness}\end{center}

\begin{itemize}
\item
  The chapter finishes with the idea that scientists, no matter what
  they say, haven't used induction in a lot of instances. Newton said
  that hypthesising is being merely speculative, but that's exactly what
  he did!
\item
  \begin{quote}
  A major problem with Newton's account of his own discoveries was
  famously pointed out by the historian and philosopher of science
  Pierre Duhem (1861--1916), namely that Kepler's laws say that the
  planets move in perfect ellipses around the Sun, but because each
  planet exerts a gravitational force on all the others and the Sun
  itself, Newton's own law of gravitation predicts that the paths of the
  planets will never be perfect ellipses. So Newton can hardly have
  inferred his laws directly from Kepler's if the latter are actually
  inconsistent with the former
  \end{quote}
\item
  There's an interesting paragraph. And one thing that confused me is
  that I don't know if someone concerned with morality would necessarily
  say apriori they would not believe torture `good' if they could not
  refute it somehow.

  \begin{itemize}
  \item
    \begin{quote}
    Given this, it is clear that, as in other areas of philosophy, we
    need to reach what is known as a `reflective equilibrium' between
    our pre- philosophical beliefs and the results of philosophical
    inquiry. Con- sider the following analogy; in ethics we inquire into
    questions about the nature of the good and the general principles
    that will guide us in trying to resolve controversial moral issues,
    such as abortion and euthanasia. However, ethicists would reject any
    ethical theory that implied that the recreational torturing of human
    beings was morally acceptable, no matter how plausible the arguments
    for it seemed. In ethics we demand that accounts of the good do not
    conflict with our most fundamental moral beliefs, although we will
    allow them to force us to revise some of our less central moral
    views. So it is with the philosophy of science; accounts of the
    scientific method that entail that those scientists who produced
    what we usually take to be the best among our scientific theories
    were proceeding in quite the wrong way will be rejected, but we will
    allow that an account of the scientific method can demand some
    revisions in scientific practice in certain areas. Indeed, it is
    permissible that we might conclude that most current science is
    being done very badly, or we might even conclude that most
    scientists are bad scientists; nonetheless, we ought not to conclude
    that our best science is bad science.
    \end{quote}
  \end{itemize}
\end{itemize}

\end{document}
