\documentclass[11pt]{article}

% basic packages
\usepackage[margin=1in]{geometry}
\usepackage{fancyhdr}
\pagestyle{fancy}

\renewcommand{\sectionmark}[1]{\markright{\textsf{\arabic{section}. #1}}}
\renewcommand{\subsectionmark}[1]{}
\setlength{\headheight}{14pt}

\linespread{1.03} % give a little extra room
\setlength{\parindent}{0.2in} % reduce paragraph indent a bit
\setcounter{secnumdepth}{2} % no numbered subsubsections
\setcounter{tocdepth}{2} % no subsubsections in ToC

\begin{document}

% make title page
\thispagestyle{empty}
\bigskip \
\vspace{0.1cm}

\begin{center}
{\fontsize{22}{22} \selectfont Notes on}
\vskip 16pt
{\fontsize{36}{36} \selectfont \bf \sffamily Power Systems}
\vskip 24pt
\end{center}


% make table of contents
\newpage
\microtoc
\newpage

% main content

\section{Context}


\textit{Build up historical context to see why Electric grid is the way it is}

Power system is a network of components deployed to supply, transfer and use
power. In particular, electric power.

The transformer allows the use of realy high voltages in generation and
transmission of Power.

Why do we need large voltages for efficient power transmission?


Initially start out with how to provide light with electricity rather than gas.
War of the currents: AC vs DC.

\section{Components of Electric Power System}

\subsection{Electric Generator}

Most electricity is generated using a turbo generator. Mechanical energy,
derived from fossil fuels spins a turbine rotor that induces power into the
surrounding stator.

A \textbf{generator} consists of a rotating part and a stationary part which
together form a magnetic circuit.

\begin{itemize}
    \item An n pole strator where each pole is a solenoid, that operates like a
        bar magnet.
    \item The strator can be connected to a power source in such a way that a
        rotating magnetic field is produced.
    \item The voltage from this power source are at different phases.
    \item Detail how this rotating magnetic field is generated.
    \item Product of superposition of magnetomotive force.

    \item For now, if we just think of the rotor as having a magnetic moment
        that is trying to follow the rotating magentic field.
\end{itemize}

Equivalent circuit enables us to analyse performance of motor.
Going down the rabbit hole of slip frequency and all that.

\subsection{Load}

Power systems deliver energy to \textit{loads} which form a function

\subsection{Transmission}

Conductors carry power from source to load.

\section{Foundation}


Generators generate sinusoidal output (natural rotation).
We can turn the rotation of coil of wire into a sinusoidal wave.

If we look at a loop in a uniform magnetic field.

Comes directly from translating mechanical energy (through rotation) to
electrical energy.

RMS value
\[
P_{DC} = P_{AC}
\] 
\[
I^2R = \frac{1}{2}I_{max}^2R
\] 
\[
I_{DC} = \sqrt{\frac{1}{2}I_{max}^2} 
\] 
\[
I_{DC} = \frac{I_{max}}{\sqrt{2} }

.\] 

For a sinusoidal wave. Think you've to integrate for other funky waveforms

From Faraday's law we know that the emf generated in a coil is proportional to
the magnetic flux. If you've a rotating coil, the flux is change by some
proportion to the angular velocity. 

Things like impedance, reactance are just products of the alternating current.

\subsection{Magentic Circuits}

How is power transferred to a load in this circuit?

\subsection{Phasors}

One of the main reasons for using AC (according to Halliday) is that the
magnetic field is changing, meaning we can use transformers to step up or step
down current.


\end{document}
