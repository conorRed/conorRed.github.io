\documentclass[11pt]{article}

% basic packages
\usepackage[margin=2cm]{geometry}
\usepackage[pdftex]{graphicx}
\usepackage{amsmath,amssymb,amsthm}
\usepackage{custom}

% page formatting
\usepackage{fancyhdr}
\pagestyle{fancy}

\renewcommand{\sectionmark}[1]{\markright{\textsf{\arabic{section}. #1}}}
\renewcommand{\subsectionmark}[1]{}
\lhead{\textbf{\thepage} \ \ \nouppercase{\rightmark}}
\chead{}
\rhead{}
\lfoot{}
\cfoot{}
\rfoot{}
\setlength{\headheight}{14pt}

\linespread{1.03} % give a little extra room
\setlength{\parindent}{0.2in} % reduce paragraph indent a bit
\setcounter{secnumdepth}{2} % no numbered subsubsections
\setcounter{tocdepth}{2} % no subsubsections in ToC

\begin{document}

% make title page
\thispagestyle{empty}
\bigskip \
\vspace{0.1cm}

\begin{center}
{\fontsize{36}{36} \selectfont \bf \sffamily }
\vskip 24pt
{\fontsize{18}{18} \selectfont \rmfamily } 
\vskip 24pt
\end{center}

% make table of contents
\newpage
\microtoc
\newpage

% main content
\hypertarget{epistemology}{%
\section{Epistemology}\label{epistemology}}

\begin{quote}
Before me is a grassy green field. A line of trees marks its far edge,
which is punctuated by a spruce on its left side and a maple on its
right. Birds are singing. A warm breeze brings the smell of roses from a
nearby trellis. I reach for a glass of iced tea, still cold to the touch
and flavored by fresh mint. I am alert, the air is clear, the scene is
quiet. My perceptions are quite distinct.
\end{quote}

\begin{quote}
It seems altogether natural to believe these things given my experience,
and I think I justifiedly believed them. I believed them, not in the way
I would if I accepted the result of wishful thinking or of merely
guessing, but with justification.
\end{quote}

\begin{itemize}
\tightlist
\item
  beliefs from perception like this justified, not through some process
  of justification but inherent in the fact that they're just considered
  that way.
\end{itemize}

\begin{quote}
Being justified in believing something is having justification for
believing it. This, in turn, is roughly a matter of having ground for
believing it\ldots{}\textbf{Our justification for believing is basic raw
material for actual justified belief; and justified belief is commonly
good raw material for knowledge.}
\end{quote}

\begin{quote}
Belief justification occurs when there is a certain kind of connection
between what yields situational justification and the justified belief
that benefits from it. Belief justification occurs when a belief is
grounded in, and thus in a way supported by (or based on), something
that gives one situational justification for that belief, such as seeing
a field of green. Seeing is of course perceiving; and perceiving is a
basic source of knowledge---perhaps our most elemental source, at least
in childhood. This is largely why perception is so large a topic in
epistemology
\end{quote}

\hypertarget{decision-theory}{%
\section{Decision Theory}\label{decision-theory}}

https://plato.stanford.edu/entries/decision-theory/

\begin{itemize}
\item
  As Sean Carroll stated, mathematics is just determining the underlying
  structure between sets.
\item
  The underlying structure for a set of prospects (options) is the
  preference relation.
\item
  Defining this preference relation is key to decision theory.
\item
  \begin{quote}
  As noted above, preference concerns the comparison of options; it is a
  relation between options. For a domain of options we speak of an
  agent's preference ordering, this being the ordering of options that
  is generated by the agent's preference between any two options in that
  domain.
  \end{quote}
\item
  Preference and preference ordering being two distinct things.
\item
  You need completeness and transitivity to do an ordering.
\item
  ``Consider first an ordering over three regular options, e.g., the
  three holiday destinations Amsterdam, Bangkok and Cardiff, denoted A,B
  and C respectively. Suppose your preference ordering is A ≺ B ≺ C .
  This information suffices to ordinally represent your judgement;
  recall that any assignment of utilities is then acceptable as long as
  C gets a higher value than B which gets a higher value than A .
  \textbf{But perhaps we want to know more than can be inferred from
  such a utility function---we want to know how much C is preferred over
  B} , compared to how much B is preferred over A . For instance, it may
  be that Bangkok is considered almost as desirable as Cardiff, but
  Amsterdam is a long way behind Bangkok, relatively speaking. Or else
  perhaps Bangkok is only marginally better than Amsterdam, compared to
  the extent to which Cardiff is better than Bangkok. \textbf{This kind
  of information about the relative distance between options, in terms
  of strength of preference or desirability, is precisely what is given
  by an interval-valued utility function.} The problem is how to
  ascertain this information.''
\end{itemize}

\end{document}
